%% TODO

%% cut down on the self references, just put one at the top of the
%% methods, and so on

%% SST usage, consider Anders work (search "*")

%% need reference in Discussion for ongoing temperature work and in
%% Auxiliary Data

%% check usage for citations, are they in text etc.

%% check out formatting in references, e.g. URLs (with spaces!)

%% line spacing

%% abstract

%% figures format etc. 

% Template for PLoS
% Version 1.0 January 2009
%
% To compile to pdf, run:
% latex plos.template
% bibtex plos.template
% latex plos.template
% latex plos.template
% dvipdf plos.template

\documentclass[10pt]{article}

% amsmath package, useful for mathematical formulas
\usepackage{amsmath}
% amssymb package, useful for mathematical symbols
\usepackage{amssymb}

% graphicx package, useful for including eps and pdf graphics
% include graphics with the command \includegraphics
\usepackage{graphicx}

% cite package, to clean up citations in the main text. Do not remove.
\usepackage{cite}

\usepackage{color} 

% Use doublespacing - comment out for single spacing
\usepackage{setspace} 
\doublespacing

%% MDSUMNER PACKAGES
\usepackage{Sweave}

% Text layout
\topmargin 0.0cm
\oddsidemargin 0.5cm
\evensidemargin 0.5cm
\textwidth 16cm 
\textheight 21cm

% Bold the 'Figure #' in the caption and separate it with a period
% Captions will be left justified
\usepackage[labelfont=bf,labelsep=period,justification=raggedright]{caption}

% Use the PLoS provided bibtex style
\bibliographystyle{plos2009}

% Remove brackets from numbering in List of References
\makeatletter
\renewcommand{\@biblabel}[1]{\quad#1.}
\makeatother


% Leave date blank
\date{}

\pagestyle{myheadings}
%% ** EDIT HERE **


%% ** EDIT HERE **
%% PLEASE INCLUDE ALL MACROS BELOW


%% END MACROS SECTION

\begin{document}

% Title must be 150 characters or less
\begin{flushleft}
{\Large
\textbf{A Bayesian Approach To Light Level Geo-Location From Archival Tags}
}
% Insert Author names, affiliations and corresponding author email.
\\
Michael D. Sumner$^{1,\ast}$, 
Simon J. Wotherspoon$^{2}$, 
Mark A. Hindell$^{3}$
\\
\bf{1} Michael D. Sumner Institute of Marine and Antarctic Studies,
University of Tasmania, Hobart, Tasmania, Australia\\
$\ast$ E-mail: mdsumner@utas.edu.au
\bf{2} Simon J. Wotherspoon Institute of Marine and Antarctic Studies,
University of Tasmania, Hobart, Tasmania, Australia
\bf{3} Mark A. Hindell Institute of Marine and Antarctic Studies,
University of Tasmania, Hobart, Tasmania, Australia
%%\\
%%\bf{2} Author2 Dept/Program/Center, Institution Name, City, State, Country
%%\\
%%\bf{3} Author3 Dept/Program/Center, Institution Name, City, State, Country
%%\\

\end{flushleft}

% Please keep the abstract between 250 and 300 words
\section*{Abstract}
A detailed method for applying light level geo-location within a
Bayesian framework for location estimation is presented. We develop a
likelihood model for the changes observed in measured light level
during twilight and show how to incorporate other sources of
data. This approach provides a primary location estimate for each
twilight and utilizes all of the available data from archival tags.

% Please keep the Author Summary between 150 and 200 words
% Use first person. PLoS ONE authors please skip this step. 
% Author Summary not valid for PLoS ONE submissions.   
%%\section*{Author Summary}

\section*{Introduction}


Determining location from archival tag data, such as light level and
temperature, is an important method for the study of marine animals,
particularly those that spend time at depth and migrate over long
distances.  In general, locations derived from archival tags are not
as accurate as those from satellite tags but satellite methods do not
work underwater and the tags are considerably more expensive. The
major source of inaccuracy is simply that observed light levels are
such an indirect indicator of global location, but the problem is
compounded by analytical approaches that are statistically inefficient
and under-utilize the tag-measured data. For example, the evaluation
of estimate accuracy is often performed as a separate process to the
estimation, with reference to secondary sources, such as maps of sea
surface temperature. These problems were described by
\cite{sibert2007sml}, who provided a state-space approach for
geo-location from raw data with a random walk movement model. By
applying the general approach adopted in \cite{sumner2009}, we
present a methodology that provides two locations per day, each with
quantified precision and estimates for intermediate locations from an
integrated movement model.

%%New tag tech: robinson2009field

The technique of determining location from light levels relies on the
relationship of tag-measurable light to solar elevation, at particular
times of day. Solar elevation is the angle of the position of the sun
relative to the horizon.  ``Light level'' is not an absolute measure
and the actual relation between solar elevation and measured light
depends on the particular tag construction and calibration. However,
the daily pattern of change in solar elevation has a predictable
effect on measured light levels and methods of determing location from
this pattern are referred to as ``light level geo-location''
\cite{H94}.  There are two main approaches: ``threshold methods''
which detect signature patterns that mark particular solar events, and
``curve methods'' (or ``template methods'') which use patterns in the rate of
change of solar elevation
\cite{Welch1999,HB01,MBCG01,Ekstrom,Ekstrom2007}.  Both approaches
rely on twilight as the most informative time of day when solar motion
has the greatest influence on changes in measurable light level.

Direct astronomical measurements can be used to derive location with
an accuracy of a few kilometres or better
\cite{bowditch2002american}. The ``longitude problem'' for eighteenth
century navigation was due to the lack of precise and reliable
time-keeping machines \cite{Lon:S98}. With an accurate reference time,
longitude is easily derived from the sun's azimuth and latitude from
its elevation.  Time-keeping in miniature tags is now cheap and simple
but given that tags cannot directly measure solar elevation their
successful use is largely hindered by the ``latitude problem''. 

Light data from archival tags provide only an indirect indication of
angular solar position. Light level geo-location is only possible
during times when the sun actually rises and sets, and the factors
that degrade the estimation process tend to mostly affect the
estimation of latitude.  Longitude is generally much simpler to
determine from light levels than latitude \cite{HB01}, but both are
susceptible to a variety of problems. In terms of modelling the data
collection process they are two dependent parameters that arguably
cannot be cleanly separated.

\subsection{Light-based derivation techniques}
\label{sec:lighttech}
Light level geo-location has been used since the mid 1980's, with the
first work from a collaborative study by Northwest Pacific and NOAA
for tagging tuna \cite{SGNOAA86,HAB86}. \cite{SGNOAA86} gave a
detailed analysis on the limits on latitude determination by
depth-specific temperatures. \cite{WDRCN92} provided the first general
application of light level geo-location, shortly followed by
\cite{H94}. Together these works established the problems of latitude
determination during equinox and the use of ocean temperatures as a
way to improve otherwise problematic estimates. This largely set the
context for applications of geo-location: longitude and latitude are
estimated separately, then further work is done to correct poor
estimates. Innovations in techniques for deriving location from light
data have been relatively rare and many publications state that the
limits of light derivation have been met and must be augmented by
auxiliary data \cite{BM02,BBTW03,TBBW04,STAH05}. Published works that
explore the application of light methods in detail are
\cite{WDRCN92,Welch1999,MBCG01}.  Recently methods to extend the use
of light data in new ways have been introduced by
\cite{sibert2007sml,Ekstrom2007} and \cite{evans2008report}.

The major difference between existing methods for light-level
geo-location is that of ``fixed point'' methods that choose critical
times and ``curve'' methods that use the rate of change in light level
during twilight.  The classic and well-known description is a form of
threshold method that requires the definition of reference angles for
solar elevation (corresponding to noon, sunrise and sunset) inferred
from identifiable light level values. Based on the same principles
used by eighteenth century mariners, the time of local noon is
determined from the mid-point of an assumed symmetric light signal
(giving longitude), and the length of day is determined from the time
between sunrise and local noon (giving latitude). The classic method
effectively assumes a fixed point that is constant over a whole day,
then solves for that in the solar equations. This threshold approach
is particularly susceptible to asymmetric weather conditions between
twilight events, Solar equinoxes, the movement of the animal, and to
slight errors in the determination of sunrise, sunset and noon
\cite{HB01,Ekstrom}.

\subsubsection{Fixed point methods }
In fixed point, or threshold, methods there are a number of ways that
the times of twilights or zenith values are determined.  The major
distinctions are between ``fixed reference'', ``variable reference''
and ``reflection'' methods \cite{M01}, with the difference between
these concerning the estimate of latitude.  A fixed reference light
level is chosen as representative of the time of the definitive zenith
angle, and latitude is found by using these times in standard solar
algorithms.  Variable reference methods choose a level as a relative
value for each day. Reflection methods rely on comparision of
subsequent dawn/dusk light curves to define these times by finding the
best fit or match between them \cite{Hill2003}.  In essence, these
methods estimate latitude based on the apparent day length, and so
latitudinal estimates are particularly susceptible to error near an
equinox, when the day length is the same at every latitude.

\subsubsection{Curve methods}
The precise rate at which the sun appears to rise or set is a function
of latitude.  Where fixed point methods effectively estimate latitude
from the apparent day length, curve methods estimate latitude from the
apparent rate of change of light level.

Published accounts of curve methods are confined to the works of
\cite{Ekstrom}, \cite{MBCG01} and \cite{sibert2007sml}. Importantly,
these methods provide a means for better utilizing the available light
data and allowing for atmospheric effects. The advantages of the curve
approach depends on three major facets: it uses the rate of change of
solar elevation over time, prescribes a limit to be applied to the
range of angles used based on physical principles, and easily accounts
for effects caused by slow variations in weather.  Curve methods use
the actual series of recorded light levels and its relationship to
solar angles.  This uses more of the actual light data, and is less
susceptible to movement of the tag between the twilight periods and
the equinox problem described above \cite{Ekstrom}.

%% further difficulties
Curve methods are still prone to physical problems that disturb the
relationship between solar elevation and measured light that would
otherwise provide a solution built purely on principles of astronomy
and atmospheric physics \cite{Ekstrom2007}. The animal's movement and
behaviour during twilight, and attenuation by clouds and water depth
disturb the relation between ambient light level and solar elevation
that are difficult or impossible to account for directly.  Further
problems such as moonlight, atmospheric effects and depth attenuation
remain incompletely explored, with the work of
\cite{Ekstrom2007,sibert2007sml} and
\cite{evans2008report}\footnote{See the abstract by Hartog et al. in
  \cite{evans2008report}.} providing directions for further work.  The
actual measured response of the tag to light, data discretization, or
further processing depend on the particular construction and design of
any given tag, presenting even more variations that need to be
accounted for in models used to derive location. This complex of
problems requires a very general and flexible approach that can be
applied to a wide range of tags and data collection scenarios.


\subsection{Aims of this study}
%% Hindell: [You haven't actually given hard numbers regarding
%% location accuracies - perhaps briefly summarise them, and then
%% point out that they can be improved]


There is considerable room for improvement in existing light-level
geo-location estimates, and in particular, from methods that separate
the derivation of location from the estimation of model uncertainty
fail to fully utilize the available information. We apply a version of
a curve method within the Bayesian framework introduced in
\cite{sumner2009}.

The key component provides the likelihood of predicted light levels
given the measured light levels. A given location $x$ at time $t$ has
a known sequence of solar elevation providing the predicted light
levels via a tag calibration. As in the previous work we augment the
light data with auxiliary environmental data to limit the range of
locations by comparison to independent environmental databases. A
behavioural model links the estimates temporally in a physically
realistic way.

The following assumptions apply to our application of light level
geo-location.

\begin{enumerate}
\item{The construction of a relationship between solar elevation and
    light depends upon the existence of the calibration
    data.} \item{This works on (log) light measured at or near the
    surface, allowing for an additive offset to account for general
    cloud conditions and ignoring the need for depth
    correction.}\item{Twilight periods are identified from the raw
    data (ensuring that twilight occurs within the period is
    sufficient) and are assumed as inputs to the model.}
\end{enumerate}
% \item{ The conditions affecting light level that are
%     not due to changes in solar elevation are constant during each
%     twilight. }


%% discussion?
These assumptions will not work for all tags, but they provide a
relatively simply application to illustrate the overall approach.
Indeed, implementations must be very tag-specific as there are a wide
variety of models, with different light responses, quantization errors
and onboard processing.  The key component is a likelihood function
that can be defined for the data set and alternative methods, such as
those by \cite{Ekstrom2007} or \cite{sibert2007sml}, or methods
relying on measurement of the length of twilight could be used. We
want to encourage a modular approach to the problem in which various
components can be deployed from related studies as appropriate for
particular species, environments or tags.


%%Summary of the paper:
We present the required components of the approach: a function to
calculate solar elevation from position and time, a calibration
function relating measured light-level to solar elevation, a lookup
function to provide a time-specific mask location and a behavioural
model. We present two examples of archival tags from different
species: a southern elephant seal and a Subantarctic fur seal.

%%% advantages to expand on here, in results and discussion
The key advantages of the approach presented here are as follows.

\begin{enumerate}
\item{The method provides two independent locations per day}
\item{Each location has an individual error estimate.}
\item{Estimation is less susceptible to equinox problems due to
    independence from calculated daylength and ability to incorporate
    non-light data.}
\item{The framework can admit any information that is available so
    that periods in which the light data are poor or missing are
    automatically augmented by the other data sources.}
\item{The approach is open and fairly simple to modify or update so remaining
    problems can be addressed by inspecting the contribution of each
    component to the result and modifying the application as
    required.}
\end{enumerate}





% Results and Discussion can be combined.
\section*{Results}

For the fur seal example, only one twilight period is available during
the afternoon on the 10 June 1999 at the release site at Amsterdam
Island. For the elephant seal example, there was one appropriate
twilight at the start of the trip (28 January 2002), and three at the end
of the of the trip at Macquarie Island (25-26 September 2002). The solar
elevation range for both examples was set at [-8, 5] chosen from short
test runs. To exclude light data outside of this range, the $\sigma$
term is increased by an order of magnitude.

The fur seal example was run for 1.4 x $10^{5}$ iterations, the
elephant seal example was run for 7.8 x $10^{5}$ iterations. For both
the first 1 x $10^{4}$ samples were discarded to allow for burn-in.

The estimates over time for longitude and latitude are shown in
Figure~\ref{fig:rawlonlat_AI} for the Subantarctic fur seal and in
Figure~\ref{fig:rawlonlat_MI} for the southern elephant seal. These
estimates consist of two independent locations per day, with
confidence intervals for each. The posterior mean value is plotted
(red) with the 95\% confidence interval. There is no change in the overall value of the location
precision at times of equinox for either example.



The range (in kms) of each estimate is plotted for longitude and
latitude in Figure~\ref{fig:locationPrecisionMI} and
Figure~\ref{fig:locationPrecisionAI}.

Figure~\ref{fig:timeSpentZ_MI} and Figure~\ref{fig:timeSpentZ_AI} show
the time spent maps derived from the entire posterior for the
intermediate locations. The fur seal has travelled directly to the
east, taking time to slowly move to the north. No data is available
for this after 19 July 1999. The fur seal travels from Macquarie Island
directly to the south-west to the coast of Antarctica at 140$^o$ E
where it remains for several weeks when it moves further west. It then
returned back to the east and north of Macquarie Island to the
shallower waters of the Campbell Plateau, remaining for several weeks
and then returning directly to Macquarie Island.

% \begin{figure}
%   \begin{center}
%     <<fig=TRUE,echo=FALSE,print=FALSE>>=
  
    % f1 <- function(x) contourLines(as.local.pimg(x), levels = quantile(x$image[x$image > 0], 0.95))
    % f2 <- function(x) {
    %   x <- as.local.pimg(x)
    %   xsub <- rowMeans(x$z) > 0
    %   ysub <- colMeans(x$z) > 0
    %   x <- list(x = x$x[xsub], y = x$y[ysub], z = x$z[xsub, ysub])
    %   x <- project(as.matrix(expand.grid(x$x, x$y)), "+proj=laea +lon_0=85 +lat_0=-35 +units=km")

    %   c(diff(range(x[,1])), diff(range(x[,2])))
    % }

    % load("G:\\backup\\Elements\\Azimuth\\DATA\\mcmcGeo\\99AT265-J261\\round5\\readyRun.Rdata")
    % library(tripEstimation)
    % lon.min <- 75
    % lon.max <- 95
    % lat.min <- -40
    % lat.max <- -30

    % tm <- tapply(d$gmt, d$segment, mean)
    % tm <- c(min(tm) - 3600, tm)


    % img <- pimg.list(tm, c(75, 95), c(-40, -30), c(200, 200), Z = FALSE)


    % xfiles <- list.files("G:/backup/Elements/Azimuth/DATA/mcmcGeo/99AT265-J261/round5", full.names = TRUE, pattern = glob2rx("X*.bin"))
    % imX <- chunk.bin(xfiles[8], img)
    % for (i in 1:7) imX <- chunk.bin(xfiles[i], imX)

    % imX <- imX[-1]
    % library(rgdal)

    % res1 <- lapply(lapply(imX, f1), function(x) project(cbind(x[[1]]$x, x[[1]]$y), "+proj=laea +lon_0=85 +lat_0=-35 +units=km"))
    % ans1 <- do.call("rbind", lapply(res1, function(x) apply(x, 2, function(x) diff(range(x)))))

    % #f1 <- function(x) contourLines(as.local.pimg(x), levels = quantile(x$image[x$image > 0], 0.85))
    % #res2 <- lapply(lapply(imX, f1), function(x) project(cbind(x[[1]]$x, x[[1]]$y), "+proj=laea +lon_0=85 +lat_0=-35 +units=km"))
    % #ans2 <- do.call("rbind", lapply(res2, function(x) apply(x, 2, function(x) diff(range(x)))))

    % ans2 <- do.call("rbind", lapply(imX, f2))

    % plot(rbind(ans1, ans2), pch = ifelse((1:156) > 78, 1, 16), xlab = "longitude CI (km)", ylab = "latitude CI (km)")


%     @

%   \end{center}
  
%       \caption{Precision of longitude and latitude values summaries from the posterior,
%       plotted with the 95\% confidence intervals (closed circles) and the
%       85\% confidence intervals (open circles) for the fur seal example. }
%     \label{fig:locationPrecisionAI}
%   end{figure}
  
  


\subsection*{Discussion}

In this work we demonstrate a systematic method for determining
location from archival tag data, with light levels as the primary data
source.  These results present rich location estimates for two
different diving animals from data recorded by attached archival
tags. The variance of location estimates is readily determined from
the posterior by calculating appropriate confidence intervals.  The
posterior estimates represent a starting point for further analysis:
confidence intervals may be calculated from a chosen percentile as
appropriate for a given model. This approach represents a significant
improvement for movement models.

The estimation of intermediate as well as primary locations provides a
way of applying measures of behavioural resources to location
estimates. Here we have presented full-trip time spent estimates as
summaries, though the distinction between primary and intermediate
locations can be used to analyse sections of trips in greater detail.

For the Subantarctic fur seal example we did not apply any
environmental data, as the animal never dives to a depth that would
provide any limit in the surrounding bathymetry, and the difficulties
with temperature matching tag and existing SST data sets overwhelmed
any clear benefit. These difficulties are the focus of on-going work
and have been described in **Anderson etc..

Future improvements could include a more topological constraint for
complicated coastlines and environmental time series that can account
for non-sensible paths that are otherwise unable to be detected by the
methods used here. The approach presented here provides a strong
foundation for improving existing methods and developing new methods
for light-based geo-location techniques.


%% \section{Conclusion}
We have demonstrated that light data can be used to greater potential
than existing methods by incorporating it in a Bayesian approach,
applying environmental data sets and and integrated movement
model. This work presented the detail and requirements for performing
light level geo-location as a component of the general framework
presented in earlier work \cite{sumner2009}.  The approach was
demonstrated with two example tag data sets, one from a Subantarctic
fur seal and one from a southern elephant seal. The software used for
these examples is freely available as the \emph{tripEstimation}
package developed by the authors \cite{tripEstimation}. The examples
presented here show time spent estimates of entire trips from
intermediate locations as well as primary location estimates with
integrated measures of accuracy derived from the posterior. The
posterior is represented by large databases of MCMC samples for each
primary and intermediate estimate.


% You may title this section "Methods" or "Models". 
% "Models" is not a valid title for PLoS ONE authors. However, PLoS ONE
% authors may use "Analysis" 
\section*{Methods}

We illustrate the approach with data from two archival tags. The first
from a Subantarctic fur seal (\emph{Arctocephalus tropicalus}) from
Amsterdam Island (37$^o$ 55'S, 77$^o$ 30'E) using data collected with
a Wildlife Computers Time-Depth-Recorder (TDR; MK7) from 10 June 1999
to 19 July 1999. The data for this tag was kindly provided by Gwen
Beauplet and its collection is described in \cite{Beauplet2004}.  The
second tag is from a southern elephant seal (\emph{Mirounga leonina})
from Macquarie Island (158$^o$ 57'E, 54$^o$ 30'S) using data collected
with a MK7 TDR from 28 January 2002 to 26 September 2002. Data
collection methods for this tag are described in \cite{BHMS02}.

This approache uses four broad types of data for determining location. 
\begin{enumerate}
\item Prior knowledge which prescribes a maximum range for each animal's
trip. 
\item  The primary location data based on the light level time
series for each twilight. 
\item Auxiliary environmental data which define the ocean environment
  or sea surface temperature as a limitation to the animal's range.
\item A movement model applied to the representation of primary
  locations $x$ and intermediate locations $z$ that link the estimates
  temporally \cite{sumner2009}.
\end{enumerate}

Incorporating the contribution of light data to the likelihood requires
the following components.

\begin{description}
\item[Solar Elevation Function]The solar elevation function $\theta(x,
  t)$ determines the elevation $\theta$ of the sun when observed from
  location $x$ at time $t$, and is determined by standard astronomical
  formulae \cite{M91}.
\item[Calibration Function]The calibration function $l(\theta)$
  relates the light level $l$ recorded by the tag for a given solar
  elevation $\theta$. The calibration function can be provided by the
  tag manufacturer, or determined by applying semi-parametric
  regression techniques to light levels recorded by the tag at a fixed
  location.
\item[Likelihood Function]The likelihood function $p(l|x,t)$
  is the probability of observing a light level $l$ at a given
  location $x$ and time $t$. In combination, the solar elevation and
  calibration functions allow an expected light level $\hat{l}$ to be
  determined for a given location and time. Given the expected light
  level, the probability of the observed light level is then
  determined based on the properties of the tag.
\end{description}

The primary location data $y$ is tag light data informing the
locations $x=\{x_{1},x_{2},\ldots,x_{n}\}$ of the tag at a sequence of
times of twilight $t=\{t_{1},t_{2},\ldots,t_{n}\}$.  For each
twilight, the light data is extracted for the period in which the sun
is rising or setting. This yields a sequence of time series, one time
series $l_{i}=\{l_{i1},l_{i2},\ldots,l_{im}\}$ for each twilight,
where $l_{ij}$ is the light level (corrected for depth) recorded at
time $t_{ij}$ within the twilight period.

% Any uniform dampening of light levels can be express as additive because of log
% measures, we apply a general smoother in a semi-parametric model - see
% Adelchi, and Silverman - GAM]

%  Assuming the tag has a logarithmic
% response \cite{H94} a generalized additive model \cite{Wood2004} is
% used to fit a smooth profile to the light data. The GAM allows an
% additive offset for each twilight to allow for different weather
% conditions.

The contribution $p(l_{ik}| x_{i}, t_{ik})$ to the total likelihood is determined
by assuming the log corrected light levels are Normally distributed
\begin{displaymath}
  \log l_{ij} \sim \operatorname{N}
  \left (\log l(\theta(x_{i},t_{ij}))+k_{i},
    \sigma^{2}f
  \right ),
\end{displaymath}
where $\theta(x,t)$ is the Sun's elevation at location $x$ and time
$t$, and $k_{i}$ is a constant to allow for attenuation due to cloud.
The solar position algorithms of \cite{M91} were encoded as \emph{R}
functions in the package \emph{tripEstimation} \cite{tripEstimation}.
For efficiency the location-independent components of solar
declination and hour angle are first precomputed from $t$ prior to
simulation.  Following \cite{Ekstrom}, a limit is applied to the
elevations used by weighting the distribution for values outside of a
given range.

The posterior is approximated with Markov Chain Monte Carlo (MCMC)
using a block update Metropolis algorithm as in previous work
\cite{sumner2009}.  An individual proposal provides solar elevations
for each $t_{ij}$ using $\sigma(x)$, and from these the calibration
function provides predicted light levels offset by attenuation
$k_{i}$. The likelihood of predicted light levels are then calculated
according to the equation above, providing the basis for estimation.

An illustration of the effects of different locations on the expected
solar elevation (and thus expected light) is shown in
Figure~\ref{fig:locations}. The matching solar elevation plots for
locations along both curves at particular locations during a morning
are shown in Figure~\ref{fig:elevations1} (squares) and
Figure~\ref{fig:elevations2} (triangles) respectively. These figures
illustrate the variation in solar elevation for different locations at
a given time, and how subtle the changes can be.

The relative line thickness in
Figures~\ref{fig:elevations1}~and~\ref{fig:elevations2} matches the
relative point sizes for the locations in
Figure~\ref{fig:locations}. There are very small difference in the
pattern of solar elevations for relatively large geographical
differences along the axis of the morning twilight between light and
dark. This axis, and its positions later in the day are represented by
gradations in the background colour.

The band of variation for the pattern of solar elevation around
sunrise\footnote{Here and elsewhere ``sunrise'' and ``sunset'' simply
  refer to their everyday meanings of ``general time of day'' at which
  the sun rises and sets. The ``twilight periods'' are defined as the
  time of day when there is a reliably recognizable change in the
  light levels---the input data do not need to be exactly classified
  as long as the period between dark and light is captured.}  is very
narrow, matching the relative ``twilight band'' that is represented by
coloured regions in Figure~\ref{fig:locations}. Different choices of
location along the sunrise great circle provide very little variation
in expected pattern of solar elevation. The pattern for locations
along the corresponding but diametrically opposite ``evening twilight
band'' provide a lot more differentiation. This is an alternative way
of viewing the narrow ``twilight region'' of information that is
available during the morning and evening. The orientation for each of
the twilight bands switches daily, which for this time match the
intersecting lines of triangles and squares in
Figure~\ref{fig:locations}.

This provides the basis for calculating the likelihood for a given
location, by comparing the difference between measured and predicted
light levels.  We use the lognormal density as described in
\cite{sumner2009}. Other methods could include simply the absolute
difference or a correlation measure, and the data itself could be
subjected to smoothing or curve-fitting of various kinds as necessary.
%% here

\subsection*{Auxiliary data} For each segment, any proposal locations
that do not fall within the prescribed range given by environmental
data are simply excluded. For efficiency, these ranges are
pre-calculated and made available as a simple lookup function for
location at a given time. This masking approach provides a very
conservative usage of these data, to avoid potentially erroneous
assumptions. We used a topographic data set to limit the estimation to
areas of ocean for both examples, as well as Reynolds OISST for the
elephant seal. Details on the application of these environmental data
sets was provided in \cite{sumner2009}. **Anderson

\subsection*{Movement model} The model also includes a behavioural
constraint. The movement model is chosen so that the mean speeds
between successive locations are independently log Normally
distributed.

\subsection*{Estimation}
MCMC requires a starting location for each primary location $x$ along
the track. This is done using a simplistic grid search to determine
the approximate maximum likelihood estimates. The likelihood is
calculated at each point of a coarse grid of locations, using a fixed
value for the attenuation $k_{i}$.  Locations that are not consistent
with the prior or masks are excluded.  Taking the mid-point of the
intersection of the current, previous and next regions for each grid
provides a location for each segment that is consistent with the light
data and the movement of the animal. Note that this result will vary
for different values of $k_{i}$ and ignores the behavioural model.

While this is a very coarse method for initialization, this gives
rough ``first pass'' locations that can seem quite reasonable.  MCMC
is in general not dependent upon the actual starting points used
\cite{Gilks:MCMC} and in theory this only affects the time taken for
eventual burn-in \footnote{``Burn-in'' is the early period of
  MCMC samples that are transient and unrepresentative of the
  equilibrium distribution.}. However, this initialization can be
problematic as the behavioural model does not account for the topology
of the prior space where accepted dog-legs paths may imply an
otherwise impossible journey due to the wide variety of coastline
shapes, or other irregular spaces implied by the masks.  Also, these
initial points are of course not the final estimates, which must be
derived from samples from the posterior after running the full
estimation model and ensuring good mixing of the chains.

The MCMC estimation proceeds by sampling from these starting locations
accepting or rejecting proposal locations as detailed in
\cite{sumner2009}.  Results are generated by direct summarization of
the posterior, by binning the samples for $x$ and $z$.  From these
binned estimates, quantiles were calculated for the binned primary $x$
estimates to give ``most probable'' locations and precision, and time
spent maps generated from the binned intermediate $z$ estimates. A
simple density estimate on each segment allow us to determine a
percentile range. For calculating the location precision of estimates
we first project the longitude-latitude coordinates to a local
instance of the Lambert Azimuthal Equal Area projection
\cite{evenden1990cartographic} to simplify the distance calculations.



% Do NOT remove this, even if you are not including acknowledgments
\section*{Acknowledgments}

% The bibtex filename
\bibliography{H:/ref/mdsRefs}

\section*{Figure Legends}
%\begin{figure}[!ht]
%\begin{center}
%%\includegraphics[width=4in]{figure_name.2.eps}
%\end{center}
%\caption{
%{\bf Bold the first sentence.}  Rest of figure 2  caption.  Caption 
%should be left justified, as specified by the options to the caption 
%package.
%}
%\label{Figure_label}
%\end{figure}


%%[MAH: combine raw lon/lat plots into a single figure]
\begin{figure}[htbp]
  \begin{center}
\includegraphics[scale=0.5]{figures/rawLonLat_MI.pdf}
\end{center}
\caption{
{\bf Mean longitude and latitude values from the posterior,
  plotted with surrounding 95\% confidence intervals for elephant seal
  example.}
 }
  \label{fig:rawlonlat_MI}
\end{figure}


\begin{figure}[htbp]
  \begin{center}
\includegraphics[scale=0.5]{figures/rawLonLat_AI}
\end{center}
\caption{
{\bf Mean longitude and latitude values from the posterior,
  plotted with surrounding 95\% confidence intervals for fur seal
  example.}
}
  \label{fig:rawlonlat_AI}
\end{figure}

\begin{figure}[htbp]
  \begin{center}
%\includegraphics[scale=0.5]{figures/locationPrecisionAI}
\includegraphics[scale=0.5]{figures/Ch4_AI}
\end{center}
\caption{
{\bf Mean longitude and latitude values from the posterior,
  plotted with the 95\% confidence intervals (closed circles) and the
  full binned estimate boundary (open circles) for the fur seal
  example. }
}
  \label{fig:locationPrecisionAI}
\end{figure}

\begin{figure}[htbp]
  \begin{center}
\includegraphics[scale=0.5]{figures/locationPrecisionMI}
\end{center}
\caption{
{\bf Mean longitude and latitude values from the posterior,
  plotted with the 95\% confidence intervals (closed circles) and the
  full binned estimate boundary (open circles) for the elephant seal
  example. }
}
  \label{fig:locationPrecisionMI}
\end{figure}



\begin{figure}[htbp]
  \begin{center}
\includegraphics[scale=0.5]{figures/timeSpentZ_MI}
\end{center}
\caption{
{\bf Time spent estimate for entire trip derived from all posterior
  samples for Macquarie Island elephant seal.}
}
  \label{fig:timeSpentZ_MI}
\end{figure}





\begin{figure}[htbp]
  \begin{center}
\includegraphics[scale=0.5]{figures/timeSpentZ_AI}
\end{center}
\caption{
{\bf Time spent estimate for entire trip derived from all
  posterior samples for Amsterdam Island fur seal.} The zonally
  (east/west) oriented lines in the inset are mean locations of
  Southern Ocean fronts \cite{orsi1995mea}. Only the Subantarctic
  front is seen in the main figure region.  }
  \label{fig:timeSpentZ_AI}
\end{figure}


\begin{figure}[htbp]
  \begin{center}
\includegraphics[scale=0.25]{figures/locations.png}
\end{center}
\caption{
{\bf Locations around Macquarie Island (158$^o$ 57'E, 54$^o$ 30'
  S).} Square icons indicate locations along the current twilight line,
  triangles indicate the corresponding ``evening''
  locations. Macquarie Island is at the intersection of the two
  curves.}
  \label{fig:locations}
\end{figure}

\begin{figure}[htbp]
  \begin{center}
\includegraphics[width=60mm,height=60mm]{figures/elevations1.png}
\end{center}
\caption{
{\bf Morning twilight solar elevations for locations around
  Macquarie Island from Figure~\ref{fig:locations} (squares).} The time
  at which solar elevation is 0 (sunrise) is indicated with a
  vertical line, and the horizontal lines show solar elevations -5 and
  3.}
  \label{fig:elevations1}
\end{figure}


\begin{figure}[htbp]
  \begin{center}
\includegraphics[width=60mm,height=60mm]{figures/elevations2.png}
\end{center}
\caption{
{\bf Morning twilight solar elevations for location around
  Macquarie Island from Figure~\ref{fig:locations} (triangles).} The
  time at which solar elevation is 0 (sunrise) is indicated
  with a vertical line, and the horizontal lines show solar elevations
  -5 and 3}
  \label{fig:elevations2}
\end{figure}



\end{document}

